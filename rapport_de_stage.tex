\documentclass[12pt,twoside]{report}
\usepackage[utf8]{inputenc}
\usepackage[T1]{fontenc}
\usepackage{natbib}
\usepackage[francais]{babel}
\usepackage[toc,page]{appendix} 
\usepackage{graphicx}
\usepackage{geometry}
\usepackage{xcolor}
\pagestyle{plain}
\geometry{top=2.5cm, bottom=3.5cm, left=3cm , right=2cm}
\renewcommand{\thechapter}{\Roman{chapter}}
\renewcommand{\thesection}{\Alph{section}}

\title{Génération automatique de tests}
\author{Lucas Nayet}
\date{Juin 2018}

\begin{document}

\makeatletter 
\begin{titlepage} 
\centering  
\vfill 
\begin{figure}[h]
\includegraphics[width = 70mm]{UPN-logo.jpg}
\end{figure}
\vspace{1.6cm}
{\LARGE {\@title}}\\ 
\vspace{1.6cm}
{\large \@author}\\ 
\vspace{1cm} {\large \@date}\\
\vfill
\end{titlepage} 
\makeatother

\thispagestyle{empty}
\setcounter{page}{0}
J'adresse mes remerciements à mon responsable de stage, Monsieur François Delbot, pour l'aide et le temps qu'il m'a consacré.\\
Je tiens également à remercier Monsieur Pascal Poizat, responsable de la MIAGE M1 classique.\\
J'adresse également mes remerciements à la formation MIAGE de l'UFR SEGMI qui m'a accueilli cette année pour ma première année de master et à toute l'équipe éducative.\\
Enfin, je tiens à remercier toutes les personnes qui m'ont aidé d'une manière ou d'une autre à la réalisation de ce stage.

\tableofcontents

\newpage
\color{red}
\chapter{Présentation du stage}
\color{black}
\section {Les circonstances du stage}
Étant étudiant à l'UFR Segmi de l'Université Paris Nanterre en M1 MIAGE, je devais effectuer un stage nécessaire pour la validation de mon M1 et mon passage en Master 2 MIAGE.\\ 
N'ayant malheureusement pas trouvé d'entreprise, le stage s'est donc déroulé dans mon université, à Nanterre. Mon responsable de stage fut donc un de mes enseignants, Monsieur François Delbot, qui m'a donné les différentes directives à suivre durant toute la durée du stage.
\section {En quoi consiste le stage ?}
Le stage porte sur la génération de tests en HTML.\\On part d'un fichier de référence que l'on appellera ici "modele.html" et à partir de ce dernier, nous allons générer une série de tests qui permettront à d'autres fichiers html d'être comparés avec le fichier de référence pour savoir s'ils sont valides ou non.
\section {Les pré-requis}
Pour effectuer cela, quelques préparatifs étaient indispensables.\\
Tout d'abord, l'environnement de travail : le système d'exploitation sur lequel je devais travailler était Linux. On l'obtient via la virtualisation avec un logiciel comme VirtualBox par exemple ou encore en mettant son ordinateur en DualBoot. Pour le stage, la virtualisation a été privilégiée.\\Ensuite, pour pouvoir faire le fond du stage (sur lequel on va s'attarder dans le chapitre suivant), je devais utiliser la bibliothèque Beautiful Soup.
\subsection{Installer Beautiful Soup}
Pour installer Beautiful Soup, rendez-vous dans un terminal et saisissez avant toute installation quelconque la commande suviante : \textbf{sudo apt-get update} qui permettra de mettre à jour votre système. Ensuite, vous pourrez saisir ce qui vous permettra d'installer la bibliothèque voulue soit \textbf{sudo pip install beautifulsoup4} si pip n'est pas reconnu, c'est qu'il n'est pas installé. Pour l'installer, faites la commande suivante : \textbf{sudo apt-get install python-pip} et une fois installé, refaites la commande précédente. Beautiful Soup est donc installée et prête à être utlisée.

\newpage
\color{red}
\chapter{Parser du HTML}
\color{black}
\section{Le fichier HTML de référence}
\end{document}
